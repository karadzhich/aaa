\documentclass[a4paper]{article}
\usepackage[utf8]{inputenc}% указание кодировки исходного текста
\usepackage[english,russian]{babel} %используемые языки. Активизирован последний из списка.
\usepackage{amsthm}
\newtheorem{theo}{Теорема}
\begin{document}
\begin{theo}[косинусов]
Для $a$, $b$, $c$ и углом $\alpha$, противолежащим стороне $a$, справедливо соотношение:
\begin{equation}\label{frm}
a^2 = b^2 + c^2 - 2 \cdot b \cdot c \cdot \cos\,\alpha.
\end{equation}
Квадрат стороны треугольника равен сумме квадратов двух других сторон минус удвоенное произведенние этих сторон на косинус угла между ними.
Ссылка на формулу: \ref{frm}
\end{theo}
\end{document}
